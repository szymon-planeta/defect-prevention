\subsection*{Test Driven Development}

Test-driven development is a common using approach in agile projects. In this section we want to focus on benefits and costs of using TDD that are presented in three following papers: \textit{ What are the costs and benefits of TDD?,}  \cite{costsbenefitstdd},\textit{ What Do We Know about Test-Driven Development?} \cite{whatdoweknowabouttdd} and \textit{Evaluating the efficacy of test-driven development: industrial case studies.} \cite{evaluatingefficacy}. 

In case of effectiveness TDD helps to increase low-level design quality \cite{evaluatingefficacy} and that cause prevent mistakes \cite{whatdoweknowabouttdd}. Moreover developers who more think about code and design increase the overall quality and it cause the returns of investments \cite{whatdoweknowabouttdd}. On the other hand developers are more motivated to think about quality of code and lead to finish their tasks frequently \cite{evaluatingefficacy}. Projects that are implemented in TDD are easier to maintain and develop. It cause a great level of saving in projects that are long-term \cite{evaluatingefficacy}, \cite{whatdoweknowabouttdd}. 

Test-driven development is a hight effort technique. It requires a lot of trainings and experience to develop code in that way. It's increase cost at the beginning, because programmers need time to learn and focus on new values and design \cite{whatdoweknowabouttdd}, \cite{evaluatingefficacy}.

To sum up, TDD technique increase effectiveness of low-level design and provide fast feedback about code to the programmers. At the beginning of the project it can increase cost, but it provide hight returning of investment when project will be future develop and maintain. Hight care about design and focus on requirements can cause low level of defects \cite{costsbenefitstdd}. However TDD is a technique that is hard to implement. To use TDD, company needs to do training and provide right mind-set of developers.
