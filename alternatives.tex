\section{Alternative solutions}
Defect prevention are actions that we perform to avoid introducing defects into the software. They are really important to focus on, but it is nearly impossible to prevent all of the defects and have totally defect-free code. Eventually defects can be introduced into the product, but there are still some methods that we can use to avoid them in the final product. These are defect prediction and defect removal.

\subsection{Defect prediction}
In the software engineering industry, it is believed that little amount of code contains most of the defects\cite{pred1}. Therefore, considering limited resources of quality assurance, it is crucial to recognize defective parts of code and properly allocate quality assurance resources. We can distinguish four phases of defect prevention - collecting metrics for the project, selecting which metrics will be used for prediction, classification and validation. Metrics that are commonly used for defect prediction divide in two groups - product metrics and process metrics. Product metrics are static and relate to the feature of the product at a given point of time (like LoC - Lines of Code). Process metrics relate more to the history of the product (the process of its creation), e.g. Number of Distinct Committers or Number of Modified Lines. We think that defect prediction can be really beneficial for companies. This area still has a lot of room for improvement, especially considering recent development in machine learning area. 

\subsection{Defect removal}
When the defect was introduced in the software and identified, it needs to be removed. Goal of every company is (or at least should be) to remove all of the defects before the delivery. One important metric that is worth mentioning here is Defect Removal Efficiency (DRE) - it measures ability of the team to remove defects before the release. This value is calculated as a ratio of removed defects to the total number of defects. The closer this value is to 1.0, the better. 

Defects of different origins have different levels of resolution difficulity. Defects that have their origins in requirements, design or bad fixes tend to be the most difficult to resolve\cite{capers}. They are the most frequently delivered to the customer. Different difficulty levels also have their reflection in repair rates - how long if takes to remove the defect. Usually the most problematic repairs are these, who are the hardest to reproduce\cite{capers}. Therefore it is a good idea to have the customer's environment reproduced as closely as possible to detect and remove defects before releasing.
