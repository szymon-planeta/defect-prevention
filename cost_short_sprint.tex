\subsection*{Short Sprints}

Short sprints is one of the defect prevention techniques. In this section we show the benefits and costs of this approach based on article \textit{Agile Scrum Sprint Length:
What’s Right for You? } \cite{agilesprints}.

The standard sprint length takes two weeks but according to \textit{Sprint Guide} it can be from one week to even four week long. Authors of article mentioned above pinpoint that short sprint provide the great amount of benefits. On the effectiveness side short sprint decrease level of procrastination by developers. They are more focus of achieving the goal and more motivated to meet the acceptance criteria. Moreover the researcher focus on level of testing. It assume that short sprints provide more days spend on testing and it's increase software quality. This techniques is recommended for the projects that are innovative and new, because ensure frequently feedback from a customer. Developers also carry about meeting requirements and if there are any obstacles or misunderstandings they can feedback from customer. According to presented study one week sprints provide great savings compare to four week sprints \cite{agilesprints}.

Effort of introduce the described technique depends on methodology that project is actual run.  In this paper we want to focus on project that already run in Scrum. In that case short sprint technique provide a medium effort from stakeholder. The great work have to do the Scrum Master, because is crucial to keep Scrum ceremonies (like Planning or Retrospective) short and valuable. Moreover developers have to deliver quickly and focus on work that have to be done frequently. 

To sum up, short sprints prevent procrastination and ensure frequently feedback and it cause the great cost savings compare to long sprints \cite{agilesprints}. The effort of using this technique is not very great, because the overall flow of the project stay the same.