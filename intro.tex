\section{Introduction}
During the course we learned a lot about quality. We discovered different definitions of quality stated by different researchers based on years of experience and conducted research. It is clearly not so simple to understand and define the concept of quality. It is even harder to identify what the quality is for the customer, especially if sometimes even the customer does not know that. Having said that, out of all possibilities, we think that the most intuitive quality feature of a product is that it is defect free. It was the most intuitive for us and it probably is for many other people. If we take a look at the quality dimensions of goods defined by Bergman and Klefsjo\cite{bergman} - reliability, performance, maintainability, environmental impact, appearance, flawlessness, safety and durability - we can easily conclude that defects in the product can cause losses in any of aforementioned quality dimensions. Therefore it is crucial to prevent defects from appearing. We think of defect prevention as of actions taken in order to deliver the product to the customer with as low possible number of defects as possible. Delivering product with lower number of defects can save a company a lot of money (it is cheaper to fix defects before releasing the product) and can keep customer satisfied (nobody wants to pay for fixing defects). 
