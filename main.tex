\documentclass[letterpaper, 10 pt, conference]{ieeeconf}  % Comment this line out
                                                          % if you need a4paper
%\documentclass[a4paper, 10pt, conference]{ieeeconf}      % Use this line for a4
                                                          % paper
\usepackage{paralist} 
\IEEEoverridecommandlockouts                              % This command is only
                                                          % needed if you want to
                                                          % use the \thanks command
\overrideIEEEmargins
% See the \addtolength command later in the file to balance the column lengths
% on the last page of the document



% The following packages can be found on http:\\www.ctan.org
%\usepackage{graphics} % for pdf, bitmapped graphics files
%\usepackage{epsfig} % for postscript graphics files
%\usepackage{mathptmx} % assumes new font selection scheme installed
%\usepackage{times} % assumes new font selection scheme installed
%\usepackage{amsmath} % assumes amsmath package installed
%\usepackage{amssymb}  % assumes amsmath package installed

\title{\LARGE \bf
Defect prevention}

%\author{ \parbox{3 in}{\centering Huibert Kwakernaak*
%         \thanks{*Use the $\backslash$thanks command to put information here}\\
%         Faculty of Electrical Engineering, Mathematics and Computer Science\\
%         University of Twente\\
%         7500 AE Enschede, The Netherlands\\
%         {\tt\small h.kwakernaak@autsubmit.com}}
%         \hspace*{ 0.5 in}
%         \parbox{3 in}{ \centering Pradeep Misra**
%         \thanks{**The footnote marks may be inserted manually}\\
%        Department of Electrical Engineering \\
%         Wright State University\\
%         Dayton, OH 45435, USA\\
%         {\tt\small pmisra@cs.wright.edu}}
%}

\author{Szymon Planeta and Paulina Russak% <-this % stops a space
}


\begin{document}



\maketitle
\thispagestyle{empty}
\pagestyle{empty}


%%%%%%%%%%%%%%%%%%%%%%%%%%%%%%%%%%%%%%%%%%%%%%%%%%%%%%%%%%%%%%%%%%%%%%%%%%%%%%%%
\begin{abstract}

	Defect prevention is crucial, but frequently not enoughly emphasized part of software development process. In this paper we present our interpretation of defect prevention, followed by a study of different defect prevention techniques - their pros and cons, cost and benefit analysis. Alternative solutions to avoid defects in the final product are presented. Finally, specific project is desbribed together with recommendations for defect prevention techniques.

\end{abstract}


%%%%%%%%%%%%%%%%%%%%%%%%%%%%%%%%%%%%%%%%%%%%%%%%%%%%%%%%%%%%%%%%%%%%%%%%%%%%%%%%
\section{Introduction}
During the course we learned a lot about quality. We discovered different definitions of quality stated by different researchers based on years of experience and conducted research. It is clearly not so simple to understand and define the concept of quality. It is even harder to identify what the quality is for the customer, especially if sometimes even the customer does not know that. Having said that, out of all possibilities, we think that the most intuitive quality feature of a product is that it is defect free. It was the most intuitive for us and it probably is for many other people. If we take a look at the quality dimensions of goods defined by Bergman and Klefsjo\cite{bergman} - reliability, performance, maintainability, environmental impact, appearance, flawlessness, safety and durability - we can easily conclude that defects in the product can cause losses in any of aforementioned quality dimensions. Therefore it is crucial to prevent defects from appearing. We think of defect prevention as of actions taken in order to deliver the product to the customer with as low possible number of defects as possible. Delivering product with lower number of defects can save a company a lot of money (it is cheaper to fix defects before releasing the product) and can keep customer satisfied (nobody wants to pay for fixing defects). 

\section{Techniques}
Defects can be introduced into the software product at every stage of software development process. According to Capers\cite{capers} there are five main categories of defect origins: requirements, design, source code, user document and bad fixes. There is no technique that will allow us to prevent all of the defects from being introduced. Many different methods exist, each of them is applicable in different areas and performance of the techniques depends on the application area. The best results are achieved when many different techniques are used, on every stage of software development - companies that use multi-stage defect prevention achieve the best results\cite{capers}. Out of many different defect prevention techniques, we decided to present three of them - Short sprints (SCRUM methodology), Pair programming and Test Driven Development (TDD).
 
\subsection{Short sprints}
Sprint in Scrum methodology is a cycle of work of a team that lasts one to four weeks. Sprint starts with "Sprint planning" meeting and ends with "Sprint retrospective". During the sprint, team members are attending everyday meetings called "Daily scrum". At the end of each sprint, team should be able to deliver working version of the product. Scrum is an iterative approach, when one sprint ends, next sprint starts.

Keeping short time of sprint can help in preventing defects. Short sprint time results in frequent contact with Product Owner - having frequent feedback from Product Owner can help in preventing requirements defects. Short sprints also result in smaller chunks of work - this may reduce the risk of introducing source code defects (it is easier to work on a limited part of code).   

Unfortunately some team members may feel uncomfortable - they can feel under the pressure of really frequent deliveries. Feeling stressed certainly won't improve their work. Other disadvantage might be the length of sprint planning and sprint retrospective meetings. Since sprint time is shorter, the meetings are more frequent. Time of the meetings should be scaled accordingly to the time of the sprint in order to avoid productivity reduction.

\subsection{Pair programming}
Pair programming is a method where two developers work together simultaneously on the same code at the same time. They are working on one workstation, one of the developers is writing the code and the other is just observing and reviewing the process.

Because of two programmers working on the same task, this method can lead to less defects introduced in the design and source code. Different experiences, different ways of thinking and different point of views lead to considering more solution methods and choosing better options, preventing defects from appearing while creating the source code. Programmers generally like working with this method, it can be helpful to build good relations in the team. According to the surveys, over 90\% of programmers enjoyed collaborative programming more than individual programming\cite{pp}.

Although this method has a lot of pluses, it is not perfect. First of all - there are two people working on the same code simultaneously. Even though the code can contain less defects, more resources were spend to produce it. The company should estimate if this method is actually beneficial for them. It can be really difficult to calculate if this method is profitable for the specific company/project. Another disadvantage is that some of the employees might feel stressed when someone else is observing them working, watching every single move. We personally know people who would not like to work like that. This method should not be forced on the employees. 

\subsection{Test Driven Development}
Test Driven Development is an approach in creating software, where developer starts with writing the code for the test case. Created test should fail - actual application code is not ready yet. Developer proceeds with writing actual functional code. When the code is ready and working properly, prepared test should pass. Programmer immediately knows that the code he produced is working as planned (assuming correctness of the test case).

Testing methods are good at preventing code defects\cite{capers}. TDD leads to capturing the defect immediately when it is introduced. With proper continuous integration pipeline, tests would be executed and prevent introducing defect into the codebase. This approach also helps to detect design problems in the early phase of development. Written tests can also be a part of documentation.

We think of TDD as a really beneficial and effective defect prevention approach, but sometimes it may be difficult to apply. It is hard to introduce this approach in complex, already existing legacy projects. Another hindrance is that it requires additional employees knowledge - training might be needed. 


\section{Alternative solutions}
Defect prevention are actions that we perform to avoid introducing defects into the software. They are really important to focus on, but it is nearly impossible to prevent all of the defects and have totally defect-free code. Eventually defects can be introduced into the product, but there are still some methods that we can use to avoid them in the final product. These are defect prediction and defect removal.

\subsection{Defect prediction}
In the software engineering industry, it is believed that little amount of code contains most of the defects\cite{pred1}. Therefore, considering limited resources of quality assurance, it is crucial to recognize defective parts of code and properly allocate quality assurance resources. We can distinguish four phases of defect prevention - collecting metrics for the project, selecting which metrics will be used for prediction, classification and validation. Metrics that are commonly used for defect prediction divide in two groups - product metrics and process metrics. Product metrics are static and relate to the feature of the product at a given point of time (like LoC - Lines of Code). Process metrics relate more to the history of the product (the process of its creation), e.g. Number of Distinct Committers or Number of Modified Lines. We think that defect prediction can be really beneficial for companies. This area still has a lot of room for improvement, especially considering recent development in machine learning area. 

\subsection{Defect removal}
When the defect was introduced in the software and identified, it needs to be removed. Goal of every company is (or at least should be) to remove all of the defects before the delivery. One important metric that is worth mentioning here is Defect Removal Efficiency (DRE) - it measures ability of the team to remove defects before the release. This value is calculated as a ratio of removed defects to the total number of defects. The closer this value is to 1.0, the better. 

Defects of different origins have different levels of resolution difficulity. Defects that have their origins in requirements, design or bad fixes tend to be the most difficult to resolve\cite{capers}. They are the most frequently delivered to the customer. Different difficulty levels also have their reflection in repair rates - how long if takes to remove the defect. Usually the most problematic repairs are these, who are the hardest to reproduce\cite{capers}. Therefore it is a good idea to have the customer's environment reproduced as closely as possible to detect and remove defects before releasing.

\section{Cost and benefit analysis}
of the previously mentioned techniques

\subsection*{Short Sprints}

Short sprints is one of the defect prevention techniques. In this section we show the benefits and costs of this approach based on article \textit{Agile Scrum Sprint Length:
What’s Right for You? } \cite{agilesprints}.

The standard sprint length takes two weeks but according to \textit{Sprint Guide} it can be from one week to even four week long. Authors of article mentioned above pinpoint that short sprint provide the great amount of benefits. On the effectiveness side short sprint decrease level of procrastination by developers. They are more focus of achieving the goal and more motivated to meet the acceptance criteria. Moreover the researcher focus on level of testing. It assume that short sprints provide more days spend on testing and it's increase software quality. This techniques is recommended for the projects that are innovative and new, because ensure frequently feedback from a customer. Developers also carry about meeting requirements and if there are any obstacles or misunderstandings they can feedback from customer. According to presented study one week sprints provide great savings compare to four week sprints \cite{agilesprints}.

Effort of introduce the described technique depends on methodology that project is actual run.  In this paper we want to focus on project that already run in Scrum. In that case short sprint technique provide a medium effort from stakeholder. The great work have to do the Scrum Master, because is crucial to keep Scrum ceremonies (like Planning or Retrospective) short and valuable. Moreover developers have to deliver quickly and focus on work that have to be done frequently. 

To sum up, short sprints prevent procrastination and ensure frequently feedback and it cause the great cost savings compare to long sprints \cite{agilesprints}. The effort of using this technique is not very great, because the overall flow of the project stay the same.
\subsection*{Pair Programming}

In this section we want to present the benefits and cost of using pair programming as a method of defect prevention. 
First we want to focus on costs and performance.
 According to Nosek \cite{collaborativeProgramming} there is quite hard to measure cost and performance of the pair programming compare to individual work and it has to be analyse on two aspects: programming speed and quality of software. 
 
  In the first case groups that work in pair programming completed the job 40\% more fast and more effective \cite{collaborativeProgramming} so there is need more man-hours to deliver the same tasks \cite{areTwoHeadsBetterThanOne}.  Authors of \cite{collaborativeProgramming} pinpoint that code written quickly with poor quality may provide more defects and cause delivery delays. Moreover pairs can contain developers that are expert in different fields so they can complement and support each other \cite{collaborativeProgramming} and it connects to the second aspect of cost: software quality that is also connect with effectiveness. Effectiveness of the pair programming was measured in \textit{Are Two Heads Better than One? On the Effectiveness of Pair Programming} \cite{collaborativeProgramming}. 
 Authors discover that pair programming increase overall quality of the software and helps to achieve the difficult goals: programmers working in group finish strenuous tasks that will be impossible to do them alone. Developers pinpoint also that pair programming helps increase creative work with new software products \cite{collaborativeProgramming} \cite{pairprogramming}. Surveys also show that developers fear to work in pairs but after trial time they indicate increase of enjoyment of work \cite{pairprogrammingsurvey}. And, what's even more important, they put more confident in solution that achieve during work in groups than alone \cite{pairprogrammingsurvey}. 
 
  Pair programmings is on the one hand quite easy to implement, because there is only need to  connect developers in pairs and any additional tools or methods are not required. On the other hand there are some hindrances, because some developers don't like each other or have a different work hours \cite{pairprogramming}.
 
 To sum up pair programming is a great method to develop tasks that are difficult and have to be finish quickly. Moreover working in groups increase quality of software, increase effectiveness, and decrease number of defects \cite{collaborativeProgramming}. Developers like collaborative work, but they have to be matched properly. Pair programming requires more man-working hours, but increase quality, so it's hard to say if it causes more costs overall. In general it's low effort technique, because it's not requires additional resources or trainings to put into practice.
\subsection*{Test Driven Development}

Test-driven development is a common using approach in agile projects. In this section we want to focus on benefits and costs of using TDD that are presented in three following papers: \textit{ What are the costs and benefits of TDD?,}  \cite{costsbenefitstdd},\textit{ What Do We Know about Test-Driven Development?} \cite{whatdoweknowabouttdd} and \textit{Evaluating the efficacy of test-driven development: industrial case studies.} \cite{evaluatingefficacy}. 

In case of effectiveness TDD helps to increase low-level design quality \cite{evaluatingefficacy} and that cause prevent mistakes \cite{whatdoweknowabouttdd}. Moreover developers who more think about code and design increase the overall quality and it cause the returns of investments \cite{whatdoweknowabouttdd}. On the other hand developers are more motivated to think about quality of code and lead to finish their tasks frequently \cite{evaluatingefficacy}. Projects that are implemented in TDD are easier to maintain and develop. It cause a great level of saving in projects that are long-term \cite{evaluatingefficacy}, \cite{whatdoweknowabouttdd}. 

Test-driven development is a hight effort technique. It requires a lot of trainings and experience to develop code in that way. It's increase cost at the beginning, because programmers need time to learn and focus on new values and design \cite{whatdoweknowabouttdd}, \cite{evaluatingefficacy}.

To sum up, TDD technique increase effectiveness of low-level design and provide fast feedback about code to the programmers. At the beginning of the project it can increase cost, but it provide hight returning of investment when project will be future develop and maintain. Hight care about design and focus on requirements can cause low level of defects \cite{costsbenefitstdd}. However TDD is a technique that is hard to implement. To use TDD, company needs to do training and provide right mind-set of developers.

\section{Recommendation}
In this section we want to present one project that we participated in and show what defect prevention methods will be appropriated on each step of project. The project is a new, innovative project in the startup company. At the beginning they have only overall vision of the product. Project takes two months and provide the product that will be maintain and develop in the future. 

Carrying about software quality in each company should be very important, but in that project it is a fundamental to use techniques that ensure hight quality, because of innovative attribute and a lot of unknowns. In that case the recommended defect prevention technique for requirement part is short sprint method. This method provides the frequent feedback from the customer, increase his satisfy and provide hight motivation of teams. The design part should be provided by test-driven development, because this methods puts a lot of focus on design and overall architecture. Next it's time for implementation part. This project is very concentrate on innovative and delivery time. Pair programming provide that developers develop difficult tasks fast, effective and with good quality. In presented project those values are crucial, so pair programming in implementation part will provide good results. The last testing part can also  come up with test driven development. Flow test-then-code provide situation that code is clean and easy to maintain. Developers can easel develop project in the future without introduce errors to the existing functionalities. 

\section{Conclusions}

Defect prevention is an important element of software development. In our interpretation, defect prevention focus on actions that company or developers provide to avoid introducing defects into the product. The key factor in that case is to deliver the lowest possible number of defect (because zero defects in product is impossible). Hight level of defect prevention in the organization ensure saving and customer satisfaction. 

In this paper we deeply focus on three defect prevention methods. Short sprints provide quick feedback and decrease procrastination, pair programming increase significant quality of the software and test-driven development improve software design and provide safe maintenance. All of these methods ensure situations when programmers are focus on code and provide lower level of defect in compare to project without those methods. 

\begin{thebibliography}{99}

\bibitem{capers} Jones, C., 1997, Software Quality: Analysis and Guidelines for Success (1st ed.), Thomson Learning

\bibitem{bergman} Bergman, B. \& Klefsjo, B., 2010, Quality from Customer Needs to Customer Satisfaction, Studentlitteratur

\bibitem{pp}  Williams, Laurie; Kessler, Robert R.; Cunningham, Ward; Jeffries, Ron, (2000). "Strengthening the case for pair programming", IEEE Software. 17 (4): 19–25. doi:10.1109/52.854064.

\bibitem{pred1} Boehm, B. W., \& Papaccio, P. N. (1988). Understanding and controlling software costs. IEEE Transactions on Software Engineering, 14, 14621477. doi:10.1109/32.6191.
 
\bibitem{pairprogramming} L. A. Williams and R. R. Kessler, "The effects of "pair-pressure" and "pair-learning" on software engineering education," Thirteenth Conference on Software Engineering Education and Training, 2000, pp. 59-65.
doi: 10.1109/CSEE.2000.827023

\bibitem{areTwoHeadsBetterThanOne}
T. Dybå, E. Arisholm, D. I. K. Sjøberg, J. E. Hannay and F. Shull, "Are Two Heads Better than One? On the Effectiveness of Pair Programming," in IEEE Software, vol. 24, no. 6, pp. 12-15, Nov.-Dec. 2007.
doi: 10.1109/MS.2007.158,

\bibitem{collaborativeProgramming}
John T. Nosek. 1998. The case for collaborative programming. Commun. ACM 41, 3 (March 1998), 105-108. DOI=http://dx.doi.org/10.1145/272287.272333 

\bibitem{pairprogrammingsurvey}
Williams,
L.,
Pair
Programming
Questionnaire,
1999,

\bibitem{costsbenefitstdd}
G. Brown, What are the costs and benefits of TDD?, Practicing Ruby, 2012 

\bibitem{whatdoweknowabouttdd}
F. Shull, G. Melnik, B. Turhan, L. Layman, M. Diep and H. Erdogmus, "What Do We Know about Test-Driven Development?," in IEEE Software, vol. 27, no. 6, pp. 16-19, Nov.-Dec. 2010.
doi: 10.1109/MS.2010.152

\bibitem{evaluatingefficacy}
Thirumalesh Bhat and Nachiappan Nagappan. 2006. Evaluating the efficacy of test-driven development: industrial case studies. In Proceedings of the 2006 ACM/IEEE international symposium on Empirical software engineering (ISESE '06). ACM, New York, NY, USA, 356-363. DOI=http://dx.doi.org/10.1145/1159733.1159787


\bibitem{agilesprints} 
B. Haughton: Agile Scrum Sprint Length:
What’s Right for You?, Cognizant, 2011
%\bibitem{manifesto} Beck Kent, Robert C. Martin and others, \emph{Agile Manifesto}, 2001, http://agilemanofesto.org

%\bibitem{jacobs_short} Jacobs D., Accelerating Process Improvement Using Agile Techniques,  Cross Talk, 2004

%\bibitem{basili} Basili V. [and six others]. Aligning Organizations through Measurement : the GQM+ Strategies Approach. Cham :Springer, 2014. Print.




\end{thebibliography}




\end{document}
