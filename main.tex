\documentclass[letterpaper, 10 pt, conference]{ieeeconf}  % Comment this line out
                                                          % if you need a4paper
%\documentclass[a4paper, 10pt, conference]{ieeeconf}      % Use this line for a4
                                                          % paper
\usepackage{paralist} 
\IEEEoverridecommandlockouts                              % This command is only
                                                          % needed if you want to
                                                          % use the \thanks command
\overrideIEEEmargins
% See the \addtolength command later in the file to balance the column lengths
% on the last page of the document



% The following packages can be found on http:\\www.ctan.org
%\usepackage{graphics} % for pdf, bitmapped graphics files
%\usepackage{epsfig} % for postscript graphics files
%\usepackage{mathptmx} % assumes new font selection scheme installed
%\usepackage{times} % assumes new font selection scheme installed
%\usepackage{amsmath} % assumes amsmath package installed
%\usepackage{amssymb}  % assumes amsmath package installed

\title{\LARGE \bf
Defect prevention}

%\author{ \parbox{3 in}{\centering Huibert Kwakernaak*
%         \thanks{*Use the $\backslash$thanks command to put information here}\\
%         Faculty of Electrical Engineering, Mathematics and Computer Science\\
%         University of Twente\\
%         7500 AE Enschede, The Netherlands\\
%         {\tt\small h.kwakernaak@autsubmit.com}}
%         \hspace*{ 0.5 in}
%         \parbox{3 in}{ \centering Pradeep Misra**
%         \thanks{**The footnote marks may be inserted manually}\\
%        Department of Electrical Engineering \\
%         Wright State University\\
%         Dayton, OH 45435, USA\\
%         {\tt\small pmisra@cs.wright.edu}}
%}

\author{Szymon Planeta and Paulina Russak% <-this % stops a space
}


\begin{document}



\maketitle
\thispagestyle{empty}
\pagestyle{empty}


%%%%%%%%%%%%%%%%%%%%%%%%%%%%%%%%%%%%%%%%%%%%%%%%%%%%%%%%%%%%%%%%%%%%%%%%%%%%%%%%
\begin{abstract}

	Defect prevention is crucial, but frequently not enoughly emphasized part of software development process. In this paper we present our interpretation of defect prevention, followed by a study of different defect prevention techniques - their pros and cons, cost and benefit analysis. Alternative solutions to avoid defects in the final product are presented. Finally, specific project is desbribed together with recommendations for defect prevention techniques.

\end{abstract}


%%%%%%%%%%%%%%%%%%%%%%%%%%%%%%%%%%%%%%%%%%%%%%%%%%%%%%%%%%%%%%%%%%%%%%%%%%%%%%%%
\section{Introduction}
intro here

\section{Techniques}
techniques here, pros and cons

\section{Alternative solutions}
alternative solutions

\section{Cost and benefit analysis}
of the previously mentioned techniques

\section{Recommendation}
for some company


\begin{thebibliography}{99}

\bibitem{capers} Jones, C., 1997, Software Quality: Analysis and Guidelines for Success (1st ed.), Thomson Learning

\bibitem{bergman} Bergman, B. \& Klefsjo, B., 2010, Quality from Customer Needs to Customer Satisfaction, Studentlitteratur

\bibitem{pp}  Williams, Laurie; Kessler, Robert R.; Cunningham, Ward; Jeffries, Ron, (2000). "Strengthening the case for pair programming", IEEE Software. 17 (4): 19–25. doi:10.1109/52.854064.

\bibitem{pred1} Boehm, B. W., \& Papaccio, P. N. (1988). Understanding and controlling software costs. IEEE Transactions on Software Engineering, 14, 14621477. doi:10.1109/32.6191.
 

%\bibitem{manifesto} Beck Kent, Robert C. Martin and others, \emph{Agile Manifesto}, 2001, http://agilemanofesto.org

%\bibitem{jacobs_short} Jacobs D., Accelerating Process Improvement Using Agile Techniques,  Cross Talk, 2004

%\bibitem{basili} Basili V. [and six others]. Aligning Organizations through Measurement : the GQM+ Strategies Approach. Cham :Springer, 2014. Print.




\end{thebibliography}




\end{document}
