\documentclass[letterpaper, 10 pt, conference]{ieeeconf}  % Comment this line out
                                                          % if you need a4paper
%\documentclass[a4paper, 10pt, conference]{ieeeconf}      % Use this line for a4
                                                          % paper
\usepackage{paralist} 
\IEEEoverridecommandlockouts                              % This command is only
                                                          % needed if you want to
                                                          % use the \thanks command
\overrideIEEEmargins
% See the \addtolength command later in the file to balance the column lengths
% on the last page of the document



% The following packages can be found on http:\\www.ctan.org
%\usepackage{graphics} % for pdf, bitmapped graphics files
%\usepackage{epsfig} % for postscript graphics files
%\usepackage{mathptmx} % assumes new font selection scheme installed
%\usepackage{times} % assumes new font selection scheme installed
%\usepackage{amsmath} % assumes amsmath package installed
%\usepackage{amssymb}  % assumes amsmath package installed

\title{\LARGE \bf
Defect prevention}

%\author{ \parbox{3 in}{\centering Huibert Kwakernaak*
%         \thanks{*Use the $\backslash$thanks command to put information here}\\
%         Faculty of Electrical Engineering, Mathematics and Computer Science\\
%         University of Twente\\
%         7500 AE Enschede, The Netherlands\\
%         {\tt\small h.kwakernaak@autsubmit.com}}
%         \hspace*{ 0.5 in}
%         \parbox{3 in}{ \centering Pradeep Misra**
%         \thanks{**The footnote marks may be inserted manually}\\
%        Department of Electrical Engineering \\
%         Wright State University\\
%         Dayton, OH 45435, USA\\
%         {\tt\small pmisra@cs.wright.edu}}
%}

\author{Szymon Planeta and Paulina Russak% <-this % stops a space
}


\begin{document}



\maketitle
\thispagestyle{empty}
\pagestyle{empty}


%%%%%%%%%%%%%%%%%%%%%%%%%%%%%%%%%%%%%%%%%%%%%%%%%%%%%%%%%%%%%%%%%%%%%%%%%%%%%%%%
\begin{abstract}

	Defect prevention is crucial, but frequently not enoughly emphasized part of software development process. In this paper we present our interpretation of defect prevention, followed by a study of different defect prevention techniques - their pros and cons, cost and benefit analysis. Alternative solutions to avoid defects in the final product are presented. Finally, specific project is desbribed together with recommendations for defect prevention techniques.

\end{abstract}


%%%%%%%%%%%%%%%%%%%%%%%%%%%%%%%%%%%%%%%%%%%%%%%%%%%%%%%%%%%%%%%%%%%%%%%%%%%%%%%%
\section{Introduction}
intro here

\section{Techniques}
techniques here, pros and cons

\section{Alternative solutions}
alternative solutions

\section{Cost and benefit analysis}
of the previously mentioned techniques

\subsection*{Short Sprints}

Short sprints is one of the defect prevention techniques. In this section we show the benefits and costs of this approach based on article \textit{Agile Scrum Sprint Length:
What’s Right for You? } \cite{agilesprints}.

The standard sprint length takes two weeks but according to \textit{Sprint Guide} it can be from one week to even four week long. Authors of article mentioned above pinpoint that short sprint provide the great amount of benefits. On the effectiveness side short sprint decrease level of procrastination by developers. They are more focus of achieving the goal and more motivated to meet the acceptance criteria. Moreover the researcher focus on level of testing. It assume that short sprints provide more days spend on testing and it's increase software quality. This techniques is recommended for the projects that are innovative and new, because ensure frequently feedback from a customer. Developers also carry about meeting requirements and if there are any obstacles or misunderstandings they can feedback from customer. According to presented study one week sprints provide great savings compare to four week sprints \cite{agilesprints}.

Effort of introduce the described technique depends on methodology that project is actual run.  In this paper we want to focus on project that already run in Scrum. In that case short sprint technique provide a medium effort from stakeholder. The great work have to do the Scrum Master, because is crucial to keep Scrum ceremonies (like Planning or Retrospective) short and valuable. Moreover developers have to deliver quickly and focus on work that have to be done frequently. 

To sum up, short sprints prevent procrastination and ensure frequently feedback and it cause the great cost savings compare to long sprints \cite{agilesprints}. The effort of using this technique is not very great, because the overall flow of the project stay the same.
\subsection*{Pair Programming}

In this section we want to present the benefits and cost of using pair programming as a method of defect prevention. 
First we want to focus on costs and performance.
 According to Nosek \cite{collaborativeProgramming} there is quite hard to measure cost and performance of the pair programming compare to individual work and it has to be analyse on two aspects: programming speed and quality of software. 
 
  In the first case groups that work in pair programming completed the job 40\% more fast and more effective \cite{collaborativeProgramming} so there is need more man-hours to deliver the same tasks \cite{areTwoHeadsBetterThanOne}.  Authors of \cite{collaborativeProgramming} pinpoint that code written quickly with poor quality may provide more defects and cause delivery delays. Moreover pairs can contain developers that are expert in different fields so they can complement and support each other \cite{collaborativeProgramming} and it connects to the second aspect of cost: software quality that is also connect with effectiveness. Effectiveness of the pair programming was measured in \textit{Are Two Heads Better than One? On the Effectiveness of Pair Programming} \cite{collaborativeProgramming}. 
 Authors discover that pair programming increase overall quality of the software and helps to achieve the difficult goals: programmers working in group finish strenuous tasks that will be impossible to do them alone. Developers pinpoint also that pair programming helps increase creative work with new software products \cite{collaborativeProgramming} \cite{pairprogramming}. Surveys also show that developers fear to work in pairs but after trial time they indicate increase of enjoyment of work \cite{pairprogrammingsurvey}. And, what's even more important, they put more confident in solution that achieve during work in groups than alone \cite{pairprogrammingsurvey}. 
 
  Pair programmings is on the one hand quite easy to implement, because there is only need to  connect developers in pairs and any additional tools or methods are not required. On the other hand there are some hindrances, because some developers don't like each other or have a different work hours \cite{pairprogramming}.
 
 To sum up pair programming is a great method to develop tasks that are difficult and have to be finish quickly. Moreover working in groups increase quality of software, increase effectiveness, and decrease number of defects \cite{collaborativeProgramming}. Developers like collaborative work, but they have to be matched properly. Pair programming requires more man-working hours, but increase quality, so it's hard to say if it causes more costs overall. In general it's low effort technique, because it's not requires additional resources or trainings to put into practice.
\subsection*{Test Driven Development}

Test-driven development is a common using approach in agile projects. In this section we want to focus on benefits and costs of using TDD that are presented in three following papers: \textit{ What are the costs and benefits of TDD?,}  \cite{costsbenefitstdd},\textit{ What Do We Know about Test-Driven Development?} \cite{whatdoweknowabouttdd} and \textit{Evaluating the efficacy of test-driven development: industrial case studies.} \cite{evaluatingefficacy}. 

In case of effectiveness TDD helps to increase low-level design quality \cite{evaluatingefficacy} and that cause prevent mistakes \cite{whatdoweknowabouttdd}. Moreover developers who more think about code and design increase the overall quality and it cause the returns of investments \cite{whatdoweknowabouttdd}. On the other hand developers are more motivated to think about quality of code and lead to finish their tasks frequently \cite{evaluatingefficacy}. Projects that are implemented in TDD are easier to maintain and develop. It cause a great level of saving in projects that are long-term \cite{evaluatingefficacy}, \cite{whatdoweknowabouttdd}. 

Test-driven development is a hight effort technique. It requires a lot of trainings and experience to develop code in that way. It's increase cost at the beginning, because programmers need time to learn and focus on new values and design \cite{whatdoweknowabouttdd}, \cite{evaluatingefficacy}.

To sum up, TDD technique increase effectiveness of low-level design and provide fast feedback about code to the programmers. At the beginning of the project it can increase cost, but it provide hight returning of investment when project will be future develop and maintain. Hight care about design and focus on requirements can cause low level of defects \cite{costsbenefitstdd}. However TDD is a technique that is hard to implement. To use TDD, company needs to do training and provide right mind-set of developers.

\section{Recommendation}
for some company

\section{Conclusions}

Defect prevention is an important element of software development. In our interpretation, defect prevention focus on actions that company or developers provide to avoid introducing defects into the product. The key factor in that case is to deliver the lowest possible number of defect (because zero defects in product is impossible). Hight level of defect prevention in the organization ensure saving and customer satisfaction. 

In this paper we deeply focus on three defect prevention methods. Short sprints provide quick feedback and decrease procrastination, pair programming increase significant quality of the software and test-driven development improve software design and provide safe maintenance. All of these methods ensure situations when programmers are focus on code and provide lower level of defect in compare to project without those methods. 

\begin{thebibliography}{99}

\bibitem{capers} Jones, C., 1997, Software Quality: Analysis and Guidelines for Success (1st ed.), Thomson Learning

\bibitem{bergman} Bergman, B. \& Klefsjo, B., 2010, Quality from Customer Needs to Customer Satisfaction, Studentlitteratur

\bibitem{pp}  Williams, Laurie; Kessler, Robert R.; Cunningham, Ward; Jeffries, Ron, (2000). "Strengthening the case for pair programming", IEEE Software. 17 (4): 19–25. doi:10.1109/52.854064.

\bibitem{pred1} Boehm, B. W., \& Papaccio, P. N. (1988). Understanding and controlling software costs. IEEE Transactions on Software Engineering, 14, 14621477. doi:10.1109/32.6191.
 
\bibitem{pairprogramming} L. A. Williams and R. R. Kessler, "The effects of "pair-pressure" and "pair-learning" on software engineering education," Thirteenth Conference on Software Engineering Education and Training, 2000, pp. 59-65.
doi: 10.1109/CSEE.2000.827023

\bibitem{areTwoHeadsBetterThanOne}
T. Dybå, E. Arisholm, D. I. K. Sjøberg, J. E. Hannay and F. Shull, "Are Two Heads Better than One? On the Effectiveness of Pair Programming," in IEEE Software, vol. 24, no. 6, pp. 12-15, Nov.-Dec. 2007.
doi: 10.1109/MS.2007.158,

\bibitem{collaborativeProgramming}
John T. Nosek. 1998. The case for collaborative programming. Commun. ACM 41, 3 (March 1998), 105-108. DOI=http://dx.doi.org/10.1145/272287.272333 

\bibitem{pairprogrammingsurvey}
Williams,
L.,
Pair
Programming
Questionnaire,
1999,

\bibitem{costsbenefitstdd}
G. Brown, What are the costs and benefits of TDD?, Practicing Ruby, 2012 

\bibitem{whatdoweknowabouttdd}
F. Shull, G. Melnik, B. Turhan, L. Layman, M. Diep and H. Erdogmus, "What Do We Know about Test-Driven Development?," in IEEE Software, vol. 27, no. 6, pp. 16-19, Nov.-Dec. 2010.
doi: 10.1109/MS.2010.152

\bibitem{evaluatingefficacy}
Thirumalesh Bhat and Nachiappan Nagappan. 2006. Evaluating the efficacy of test-driven development: industrial case studies. In Proceedings of the 2006 ACM/IEEE international symposium on Empirical software engineering (ISESE '06). ACM, New York, NY, USA, 356-363. DOI=http://dx.doi.org/10.1145/1159733.1159787


\bibitem{agilesprints} 
B. Haughton: Agile Scrum Sprint Length:
What’s Right for You?, Cognizant, 2011
%\bibitem{manifesto} Beck Kent, Robert C. Martin and others, \emph{Agile Manifesto}, 2001, http://agilemanofesto.org

%\bibitem{jacobs_short} Jacobs D., Accelerating Process Improvement Using Agile Techniques,  Cross Talk, 2004

%\bibitem{basili} Basili V. [and six others]. Aligning Organizations through Measurement : the GQM+ Strategies Approach. Cham :Springer, 2014. Print.




\end{thebibliography}




\end{document}
