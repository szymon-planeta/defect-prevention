\section{Conclusions}

Defect prevention is an important element of software development. In our interpretation, defect prevention focus on actions that company or developers provide to avoid introducing defects into the product. The key factor in that case is to deliver the lowest possible number of defect (because zero defects in product is impossible). Hight level of defect prevention in the organization ensure saving and customer satisfaction. 

In this paper we deeply focus on three defect prevention methods. Short sprints provide quick feedback and decrease procrastination, pair programming increase significant quality of the software and test-driven development improve software design and provide safe maintenance. All of these methods ensure situations when programmers are focus on code and provide lower level of defect in compare to project without those methods. 