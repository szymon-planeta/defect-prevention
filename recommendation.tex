\section{Recommendation}
In this section we want to present one project that we participated in and show what defect prevention methods will be appropriated on each step of project. The project is a new, innovative project in the startup company. At the beginning they have only overall vision of the product. Project takes two months and provide the product that will be maintain and develop in the future. 

Carrying about software quality in each company should be very important, but in that project it is a fundamental to use techniques that ensure hight quality, because of innovative attribute and a lot of unknowns. In that case the recommended defect prevention technique for requirement part is short sprint method. This method provides the frequent feedback from the customer, increase his satisfy and provide hight motivation of teams. The design part should be provided by test-driven development, because this methods puts a lot of focus on design and overall architecture. Next it's time for implementation part. This project is very concentrate on innovative and delivery time. Pair programming provide that developers develop difficult tasks fast, effective and with good quality. In presented project those values are crucial, so pair programming in implementation part will provide good results. The last testing part can also  come up with test driven development. Flow test-then-code provide situation that code is clean and easy to maintain. Developers can easel develop project in the future without introduce errors to the existing functionalities. 
