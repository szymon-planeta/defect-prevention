\subsection*{Pair Programming}

In this section we want to present the benefits and cost of using pair programming as a method of defect prevention. 
First we want to focus on costs and performance.
 According to Nosek \cite{collaborativeProgramming} there is quite hard to measure cost and performance of the pair programming compare to individual work and it has to be analyse on two aspects: programming speed and quality of software. 
 
  In the first case groups that work in pair programming completed the job 40\% more fast and more effective \cite{collaborativeProgramming} so there is need more man-hours to deliver the same tasks \cite{areTwoHeadsBetterThanOne}.  Authors of \cite{collaborativeProgramming} pinpoint that code written quickly with poor quality may provide more defects and cause delivery delays. Moreover pairs can contain developers that are expert in different fields so they can complement and support each other \cite{collaborativeProgramming} and it connects to the second aspect of cost: software quality that is also connect with effectiveness. Effectiveness of the pair programming was measured in \textit{Are Two Heads Better than One? On the Effectiveness of Pair Programming} \cite{collaborativeProgramming}. 
 Authors discover that pair programming increase overall quality of the software and helps to achieve the difficult goals: programmers working in group finish strenuous tasks that will be impossible to do them alone. Developers pinpoint also that pair programming helps increase creative work with new software products \cite{collaborativeProgramming} \cite{pairprogramming}. Surveys also show that developers fear to work in pairs but after trial time they indicate increase of enjoyment of work \cite{pairprogrammingsurvey}. And, what's even more important, they put more confident in solution that achieve during work in groups than alone \cite{pairprogrammingsurvey}. 
 
  Pair programmings is on the one hand quite easy to implement, because there is only need to  connect developers in pairs and any additional tools or methods are not required. On the other hand there are some hindrances, because some developers don't like each other or have a different work hours \cite{pairprogramming}.
 
 To sum up pair programming is a great method to develop tasks that are difficult and have to be finish quickly. Moreover working in groups increase quality of software, increase effectiveness, and decrease number of defects \cite{collaborativeProgramming}. Developers like collaborative work, but they have to be matched properly. Pair programming requires more man-working hours, but increase quality, so it's hard to say if it causes more costs overall. In general it's low effort technique, because it's not requires additional resources or trainings to put into practice.